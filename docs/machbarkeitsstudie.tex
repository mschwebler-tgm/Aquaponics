\documentclass[11pt]{article}
%Gummi|065|=)
\usepackage{eurosym}
\title{\textbf{Home Aquaponics System}}

\author{Samuel Schober, Konrad Kelc\\
		Matthias Schwebler, Ramin Bahadoorifar\\}
\date{}
\begin{document}

\maketitle
\begin{titlepage}
    \begin{center}
        \vspace*{1cm}
		\title{\textbf{Home Aquaponics System}}

        \vspace{2.5cm}

        2016 - 5BHIT

        \vspace*{5mm}

        \small{Samuel Schober, Konrad Kelc\\Matthias Schwebler, Ramin Bahadoorifar}

        \vspace*{2cm}

        \vfill

        \begin{center}
        \end{center}

    \end{center}
\end{titlepage}

\section{Einf\"uhrung}
Aquaponik ist ein Verfahren, welches die Aufzucht von Fischen mit der Aufzucht von Nutzpflanzen verbindet. Im Wesentlichen handelt es sich dabei um einen geschlossenen Wasserkreislauf, in dem die entsprechenden Nährstoffe „automatisch“ erzeugt werden; d.h. die Ausscheidungen der Fische werden durch Bakterien in N\"ahrstoffe f\"ur die Pflanzen umgewandelt. Es soll eine vollautomatisierte L\"osung f\"ur ein Aquaponik-System f\"ur kleine Haushalte als Prototyp geschaffen werden. Diese beinhaltet einerseits eine geeignete Konstruktion die sowohl das Aquarium enth\"alt als auch die entsprechende \"Uberwachung und Regelung des Systems.


\section{Projektdaten}

\subsection{Projetkteam}
Name: Samuel Schober \\
E-Mail: sschober@student.tgm.ac.at \\
\\Name: Matthias Schwebler \\
E-Mail: mschwebler@student.tgm.ac.at \\
\\Name: Ramin Bahadoorifar \\
E-Mail: rbahadoorifar@student.tgm.ac.at \\
\\Name: Konrad Kelc \\
E-Mail: kkelc@student.tgm.ac.at

\subsection{Projektbeschreibung}
Im Rahmen des Projekts wird ein Tool zur \"Uberwachung und Steuerung von Metadaten eines Aquaponic Systems, entwickelt und mit entsprechender Hardware realisiert. Dabei werden Sensoren von einem Raspbery Pi angesprochen und die Ergebnisse lokal gespeichert, welche sp\"ater \"uber eine Webseite eingesehen werden k\"onnen (M\"ogliche Methoden der \"Ubertragung werden im Kapitel X.x erleutert - TODO). Des Weiteren kann der Benutzer \"uber die Weboberfl\"ache die Wassertemperatur sowie die Belichtungsdauer der Pflanzen regeln. Zusätzlich erh\"alt der Benutzer Vorschl\"age f\"ur einen sinnvollen Besatz des Aquariums sowie f\"ur die Pflanzenwahl und deren Bed\"urfnisse.

\section{Voruntersuchung des Projekts}
\subsection{Ist-Erhebung}
Aquaponic Systeme sind haupts\"achlich im gr\"o{\ss}eren Ma{\ss}stab bei der Aufzucht von Nutzpflanzen im Einsatz. F\"ur den normalen Haushalt gibt es allerdings bis jetzt keine Marktf\"ahige L\"osung. Die gr\"o{\ss}ten zwei Crowdfunding Projekte sind:
\begin{itemize}
	\item \textbf{EcoQube C}\\
		Der EcoQube C vereint ein Handliches Aquaponics System mit elegantem Design, jedoch mangelt es an Konfigurierbarkeit. Es steht lediglich eine Fernbedienung zur verfügung, mit der die Farbe der LEDs gesteuert werden kann. Des Weiteren liegt das Fassungsvermögen des Aquariums weit unter 50 Liter, was zur Folge hat, dass in \"Osterreich maximal ein einziger Fisch darin gehalten werden kann. Daraus resultiert ein sehr kleines, ineffizientes Aquaponics System mit Mangel an Konfigurationsm\"oglichkeiten. \\
	\item \textbf{Grove Ecosystem}\\
	Das Grove Ecosystem ist das "Non Plus Ultra", wenn es um Aquaponic Systeme im Haushalt geht. Es bietet alle m\"oglchen Sensoren (Luftfeuchtigkeit und -temperatur sowie Wasserstand und -temperatur), welche \"uber eine App abgefragt werden k\"onnen. Diese bietet zus\"atzlich eine gro{\ss}e Ansammlung an Daten und daher Empfehlungen f\"ur m\"ogliche Fische und die dazu passenden Pflanzen. \\
	Dieses Paket ist allerdings nur in den USA und Kanada, mit einem Einstiegspreis von $>$ 4000\euro\hspace{0.5em}erh\"altlich.
\end{itemize}
Beide dieser Systeme sind in \"Osterreich kaum brauchbar bzw. nicht erh\"altlich. Bei Home Aquaponics wird Wert darauf gelegt, dass das fertige Produkt f\"ur jeden leistbar ist, indem Features weggelassen werden, welche nicht unbeding ben\"otigt werden. Au{\ss}erdem wird \"au{\ss}erst stromsparende Hardware verwendet, um so wenig monatliche Kosten wie m\"oglich zu verursachen.

\section{Aquaponic System}

\section{Sensoren}

\section{Aktoren}

\section{Single Board Computer}

\section{\"Ubermittlung u. Verwaltung der Daten}

\section{Webframework}
\end{document}
