\documentclass[12pt]{article}

\usepackage[english]{babel}
\usepackage[utf8x]{inputenc}
\usepackage{amsmath}
\usepackage{graphicx}
\usepackage[colorinlistoftodos]{todonotes}
\usepackage{listings}
\usepackage{glossaries}
\usepackage{placeins}
\usepackage{fixltx2e}
\usepackage{pdfpages}
\usepackage{lastpage}
\usepackage{enumitem}
\usepackage{xcolor}
\usepackage{scrpage2}
\usepackage{scrtime}
\usepackage{parskip}
\usepackage{hyperref} 

\clearscrheadfoot
\pagestyle{scrheadings}
\usepackage[
top    = 2.5cm,
bottom = 3cm,
left   = 3cm,
right  = 3cm]{geometry}
\setcounter{secnumdepth}{4}
\title{Diploma Thesis}


\author{RoboNav}
\date{\today}


%Includes
\setlength{\parindent}{0cm}

\newcommand{\executeiffilenewer}[3]{%
\ifnum\pdfstrcmp{\pdffilemoddate{#1}}%
{\pdffilemoddate{#2}}>0%
{\immediate\write18{#3}}\fi%
}
\newcommand{\includesvg}[1]{%
\executeiffilenewer{#1.svg}{#1.pdf}%
{inkscape -z -D --file=#1.svg --export-pdf=#1.pdf --export-latex}%
\input{#1.pdf_tex}[width=1.0\textwidth]%
}

%Bibtex
\def\BibTeX{{\rm B\kern-.05em{\sc i\kern-.025em b}\kern-.08em
		T\kern-.1667em\lower.7ex\hbox{E}\kern-.125emX}}


% html
\lstdefinelanguage{HTML5}{
  language=html,
  sensitive=true,	
  alsoletter={<>=-},	
  morecomment=[s]{<!-}{-->},
  tag=[s],
  otherkeywords={
  % General
  >,
  % Standard tags
	<!DOCTYPE,
  </html, <html, <head, <title, </title, <style, </style, <link, </head, <meta, />,
	% body
	</body, <body,
	% Divs
	</div, <div, </div>, 
	% Paragraphs
	</p, <p, </p>,
	% scripts
	</script, <script,
  % More tags...
  <canvas, /canvas>, <svg, <rect, <animateTransform, </rect>, </svg>, <video, <source, <iframe, </iframe>, </video>, <image, </image>, <header, </header, <article, </article
  },
  ndkeywords={
  % General
  =,
  % HTML attributes
  charset=, src=, id=, width=, height=, style=, type=, rel=, href=,
  % SVG attributes
  fill=, attributeName=, begin=, dur=, from=, to=, poster=, controls=, x=, y=, repeatCount=, xlink:href=,
  % properties
  margin:, padding:, background-image:, border:, top:, left:, position:, width:, height:, margin-top:, margin-bottom:, font-size:, line-height:,
	% CSS3 properties
  transform:, -moz-transform:, -webkit-transform:,
  animation:, -webkit-animation:,
  transition:,  transition-duration:, transition-property:, transition-timing-function:,
  }
}

% javascript language
\definecolor{lightgray}{rgb}{.9,.9,.9}
\definecolor{darkgray}{rgb}{.4,.4,.4}
\definecolor{purple}{rgb}{0.65, 0.12, 0.82}
\definecolor{babyblueeyes}{rgb}{0.63, 0.79, 0.95}

\lstdefinelanguage{JavaScript}{
  keywords={typeof, new, true, false, catch, function, return, null, catch, switch, var, if, in, while, do, else, case, break},
  keywordstyle=\color{blue}\bfseries,
  ndkeywords={class, export, boolean, throw, implements, import, this},
  ndkeywordstyle=\color{darkgray}\bfseries,
  identifierstyle=\color{black},
  sensitive=false,
  comment=[l]{//},
  morecomment=[s]{/*}{*/},
  commentstyle=\color{purple}\ttfamily,
  stringstyle=\color{babyblueeyes}\ttfamily,
  morestring=[b]',
  morestring=[b]"
}

\lstset{
   language=JavaScript,
   extendedchars=true,
   basicstyle=\footnotesize\ttfamily,
   showstringspaces=false,
   showspaces=false,
   numbers=left,
   numberstyle=\footnotesize,
   numbersep=9pt,
   tabsize=2,
   breaklines=true,
   showtabs=false,
   captionpos=b
}

% json language
\colorlet{punct}{red!60!black}
\definecolor{background}{HTML}{EEEEEE}
\definecolor{delim}{RGB}{20,105,176}
\colorlet{numb}{magenta!60!black}

\lstdefinelanguage{json}{
    literate=
     *{0}{{{\color{numb}0}}}{1}
      {1}{{{\color{numb}1}}}{1}
      {2}{{{\color{numb}2}}}{1}
      {3}{{{\color{numb}3}}}{1}
      {4}{{{\color{numb}4}}}{1}
      {5}{{{\color{numb}5}}}{1}
      {6}{{{\color{numb}6}}}{1}
      {7}{{{\color{numb}7}}}{1}
      {8}{{{\color{numb}8}}}{1}
      {9}{{{\color{numb}9}}}{1}
      {:}{{{\color{punct}{:}}}}{1}
      {,}{{{\color{punct}{,}}}}{1}
      {\{}{{{\color{delim}{\{}}}}{1}
      {\}}{{{\color{delim}{\}}}}}{1}
      {[}{{{\color{delim}{[}}}}{1}
      {]}{{{\color{delim}{]}}}}{1},
}



%lslisting
\lstdefinestyle{customjava}{
  	language=Java,
  	frame=tlrb,
  	aboveskip=3mm,
  	belowskip=6mm,
  	showstringspaces=false,
  	columns=flexible,
  	basicstyle={\small\ttfamily},
  	numbers=none,
  	numberstyle=\tiny\color{gray},
  	keywordstyle=\color{purple},
  	commentstyle=\color{orange},
  	stringstyle=\color{blue},
  	breaklines=true,
  	breakatwhitespace=true
  	tabsize=3,
  	captionpos=b,
}

\captionsetup[lstlisting]{aboveskip=5pt,belowskip=\baselineskip}
\lstset{escapechar=@,style=customjava}

%Cite right
\newcommand{\citeof}[2]{{
		\par \begingroup \leftskip=1cm \noindent \textit 
		''#1'' \cite{#2} \\
		\par \endgroup
	}}
	
% Picture insert (UseCase)
% \insertpicture{mik.png}{Some picture}{\cite{bk_key}}{itm:pic1}{0.5}
\newcommand{\insertpicture}[5]{{
	\begin{figure}[!htb]
		\centering\includegraphics[width=#5\textwidth]{#1}
		\caption[#2 #3]{#2}
		\label{#4}
	\end{figure}
	\FloatBarrier
}}

\usepackage{minted}
\usemintedstyle{borland}
\makeglossaries
\newglossaryentry{json} {name=JSON, description={Java Script Object Notation}}
\newglossaryentry{GPS} {name=GPS, description={Global Positioning System }}
\newglossaryentry{NDGPS} {name=NDGPS, description={National Differential Global Positioning System }}
\newglossaryentry{WADGPS} {name=WADGPS, description={Wide Area Differential Global Positioning System }}
\newglossaryentry{QR} {name=QR, description={Quick Response (Code) }}
\newglossaryentry{RNCP} {name=RNCP, description={RoboNav Communication Protocol}}
\newglossaryentry{TGM} {name=TGM, description={Technologisches Gewerbe Museum / Institute of technology}}
\newglossaryentry{API} {name=API, description={Application Programming Interface}}
\newglossaryentry{JNI} {name=JNI, description={Java Native Interface}}
\newglossaryentry{WiFi} {name=WiFi, description={Wireless Local Area Network}}
\newglossaryentry{DHCP} {name=DHCP, description={Dynamic Host Configuration Protocol}}
\newglossaryentry{IP} {name=IP, description={Internet Protocol}}
\newglossaryentry{MOV} {name=MOV, description={QuickTime Movie (file extension)}}
\newglossaryentry{MPEG4} {name=MPEG4, description={Moving Picture Experts Group 4 (file extension)}}
\newglossaryentry{MJPEG} {name=MJPEG, description={Motion Joint Photographic Experts Group (file extension)}}
\newglossaryentry{JPEG} {name=JPEG, description={Joint Photographic Experts Group (file extension)}}
\newglossaryentry{CPU} {name=CPU, description={Central Processing Unit}}
\newglossaryentry{GHz} {name=GHz, description={Gigahertz (thousands of MHz)}}
\newglossaryentry{ROS} {name=ROS, description={Robot Operationg System}}
\newglossaryentry{GNUGPL} {name=GNU GPL, description={Gnu's Not Unix General Public License}}
\newglossaryentry{BSD} {name=BSD, description={Berkeley Software Distribution}}
\newglossaryentry{GUI} {name=GUI, description={Graphical User Interface}}
\newglossaryentry{DFT} {name=DFT, description={Discrete Fourier Transform}}
\newglossaryentry{RGB} {name=RGB, description={Red Green Blue}}
\newglossaryentry{UML} {name=UML, description={Unified Modeling Language}}
\newglossaryentry{MRDS} {name=MRDS, description={Microsoft Robotics Developer Studio}}
\newglossaryentry{IDE} {name=IDE, description={Integrated Development Environment}}
\newglossaryentry{FXML} {name=FXML, description={JavaFX - Extensible Markup Language file}}
\newglossaryentry{XML} {name=XML, description={Extensible Markup Language}}
\newglossaryentry{MVC} {name=MVC, description={Model View Control}}
\newglossaryentry{CSS} {name=CSS, description={Cascading Style Sheet}}
\newglossaryentry{Webapp} {name=Webapplikation, description={Eine browserbasierte Anwendung, basierend auf dem Server-Client-Modell}}
\newglossaryentry{NoSQL} {name=NoSQL, description=Not only SQL}
\newglossaryentry{SBC} {name=SBC, description=Single Board Computer (Einplatinen-Computer)}
\newglossaryentry{RJ45} {name=RJ45, description=Der heutzutage Standard-Ethernet-Port}
\newglossaryentry{OS} {name=OS, description=Operating System (Betriebssystem)}
\newglossaryentry{GPIO} {name=GPIO, description=General Purpose Input/Output}
\newglossaryentry{USB} {name=USB, description=Universal Serial Bus}
\newglossaryentry{EC} {name=EC, description=Electrical Conductivity}
\newglossaryentry{RDBMS} {name=RDBMS, description=Relational Database Management System}
\newglossaryentry{SQL} {name=SQL, description=Structured Query Language}
\newglossaryentry{MySQL} {name=MySQL, description=Eines der weltweit verbreitetsten relationalen \gls{DBMS}}
\newglossaryentry{DBMS} {name=DBMS, description=Database Management System}
\newglossaryentry{DDL} {name=DDL, description=Data Definition Language}
\newglossaryentry{DML} {name=DML, description=Data Manipulation Language}
\newglossaryentry{DCL} {name=DCL, description=Data Control Language}
\newglossaryentry{ACID} {name=ACID, description={Atomicity, Consistency, Isolation, Durability}}
\newglossaryentry{BASE} {name=BASE, description={Basically Available, Soft State, Eventually Consistent}}
\newglossaryentry{WLAN} {name=WLAN, description=Wireless Local Area Network}
\newglossaryentry{LED} {name=LED, description=Light Emitting Diode}
\newglossaryentry{JSON} {name=JSON, description=JavaScript Object Notation}
\newglossaryentry{HTTP} {name=HTTP, description=Hypertext Transfer Protocol}
\newglossaryentry{NFT} {name=NFT, description=Nutrient Film Technique}
\newglossaryentry{USV} {name=USV, description=Unterbrechungssichere Stromversorgung}
\newglossaryentry{DWC} {name=DWC, description=Deep Water Culture}
\newglossaryentry{PC} {name=PC, description=Personal Computer}
\newglossaryentry{PWM} {name=PWM, description=Pulse Width Modulation}
\newglossaryentry{ERD} {name=ERD, description=Entity Relationship Model}
\newglossaryentry{Sketchup} {name=Sketchup, description={Ein 3-D-Modellierungsprogramm verwendet für Baupläne, Skizzen und der Gleichen}}
\newglossaryentry{MOSFET} {name=MOSFET, description=Metall-Oxid-Halbleiter-Feldeffekttransistor (metal-oxide-semiconductor field-effect transistor)}
\newglossaryentry{CSV} {name=CSV, description=Comma-separated values}
\newglossaryentry{NPM} {name=NPM, description=Node Package Manager}
\newglossaryentry{PHP} {name=PHP, description=Hypertext Preprocessor}

\renewcommand*\glspostdescription{\dotfill}




\begin{document}


\begin{titlepage}
\begin{center}

% Additional to first page
\includegraphics[width=0.7\textwidth]{images/logo}\\

\LARGE TGM - HTBLuVA Wien XX \\ IT Abteilung  \\[1.5cm]

% Title
\rule{1.0\textwidth}{1mm}
{ \huge \bfseries \\[0.4cm]  \huge Diplomarbeit \\ \LARGE Urban Green \\[0.4cm] }
\rule{1.0\textwidth}{1mm}

{ \huge Ramin Bahadoorifar \\ Matthias Schwebler \\ Samuel Schober \\ Konrad Kelc \\[0.4cm] }



\noindent 


\vfill

% Bottom of the page
{\small Version: \today ~at  \thistime    }
\end{center}

%\end{center}
\end{titlepage}


%HEADER AND FOOTER
\pagenumbering{Roman}
\ohead{\headmark}
\ifoot{© RoboNav - 2014/15}
\ofoot{\pagemark}

\newpage %----------------------------------------------------------------------------------------------
{\small\color{white}.}
\vspace{-0.7cm}
\tableofcontents

\newpage %----------------------------------------------------------------------------------------------
% Affirmation
%\includepdf[scale=1.2,pages=-]{documents/soor}
%\newpage
\section*{Abstract}
\cfoot{Ramin Bahadoorifar}

This diploma project is about the development of an automated low-cost aquaponic-system prototype in cooperation with the startup company Ponix Systems. An aqua-ponic-system is a closed ecosystem which consists of an aquarium and a flower bed on top of it. The water in the aquarium gets enriched with various minerals by fish-wastes and is pumped to the flower field. With these essential nutrients and the help of a special illumination the crops' growth time can be shortened significantly. \\ \mbox{} \\
The main goal of this project is to create an automated aquaponic product that contains various additional features to help newcomers to get familiar with the world of aquaponic. In particular we implemented sensores (pH-, EC- and temperature sensor) and actuators (fish feeder and a specialised illumination) to allow the user to automate the system as far as possible. Furthermore a touchscreen was installed to be able to show real-time data to the user and to establish a connection to the local Wifi. More specifically the sensor-data is temporarely saved to an in-memory database (Redis) on a raspberry pi to be accessed by the webapp later on. \\ \mbox{} \\
In order to create statistics and trendanalyses the data is sent to a centralised server where it gets persisted. This server also offers a webapp where all data can be reviewed and analysed by the user. Via both webapps (www and local) it is possible to monitor and controll the aquaponic system. The difference is that the centralised webapp (which can be accessed over the internet) supports visualisation of longtermdata, whereas the raspberry pi is not suitable to for this kind of feature. Moreover the user can monitor and adjust his aquaponic system via the world wide web and one doesn't have to be present to regulate parameters. \\ \mbox{} \\
The web application was developed with the innovative framework Angular 2 which uses WebSockets for communication between server and webapp. Node.js was used as a server framework both on the central server and on the aquaponic-system. \\ \mbox{} \\
By automatically creating regular backups to prevent dataloss after a power failure and saving timestamps to the corresponding data to transmit it after the internet connection gets established again, the system is safe against power- and Wifi failures. \\ \mbox{} \\
The final stage of the project is to build a prototype which fits the requirements to bridge the current gap in the market of middle europe.
\newpage
\section*{Kurzfassung}
\cfoot{Matthias Schwebler}

Dieses Diplomprojekt, wird in Kooperation mit der Firma "Ponix Systems" durchgef\"uhrt und handelt \"uber die Entwicklung eines Low-Cost Prototypen eines Aquaponik Systems f\"ur den Heimgebrauch um dem globalen Trend der Umweltverbesserung zu folgen. Bei einem Aquaponik System handelt es sich um ein in sich geschlossenes Ecosystem, in dem Nutzpflanzen mit Wasser aus einem Aquarium bewässert und zusätzlich mit spezieller Beleuchtung  behandelt werden, um eine stark reduzierte Wachstumszeit zu erzielen. Hauptziel unseres Projektes ist es, die derzeitige Marktl\"ucke von Aquaponik Systemen in Mitteleuropa zu f\"ullen. Daf\"ur wird ein pflegeleichtes \"Okosystem - bestehend aus einem Aquarium und einem Beet - entwickelt und auf die \"Uberwachung und Automatisierung optimiert. Dazu werden diverse Sensoren, Aktoren und Single Board Computer verbaut, welche die Regulierung der Parameter des Systems \"ubernehmen. Die Daten des Aquariums bzw. der Pflanzen werden an einen zentralen Server geleitet und dort persistiert um gegebenenfalls gew\"unschte Diagramme und Graphen zu erstellen. F\"ur die Persistierung wird auf Serverseite eine NoSQL Datenbank (MongoDB) verwendet und auf Seiten der Single Board Computer eine lightweight und schnelle In-Memory-Datenbank verwendet (RedisDB). 
Das Aquaponiksystem kann von einer Webapplikation aus beobachtet und gesteuert werden. Die dafür benötigten Daten werden an einem Zentralserver durch ein Socket in einem bestimmten Interval gesendet. Diese werden zeitlich angeordnet in der NoSQL Datenbank gespeichert und dem jeweiligen Benutzer zugeordnet, aus welchen man dann eine Zeitanalyse erstellen kann in Form eines Diagramms. Bei der Entwicklung der Webapplikation wurde das innovative Framework Angular 2 verwendet.
Die Verbindung des \"Okosystems zum Internet wird von einem Single Board Computer \"ubernommen, der \"uber einen kleinen Touchscreen bedient werden kann. Selbst bei Internetunterbrechungen zeichnet das System weiterhin Daten auf und sendet sie bei erneutem Internetzugriff mit korrektem Zeitstempel an den Server
\newpage
\section*{Acknowledgements}
\cfoot{}

We want to thank everyone. \\
 - Schabel (Idee, Grundstein) \\
 - Ponix Systems (Hardware, Softwareunterst\"utzung) \\
 - Koppensteiner (Bereitstellung des Arbeitsraums, Platz für Aquarium usw.) \\
TODO


\label{pageRomanEnd}

\newpage %----------------------------------------------------------------------------------------------
\pagenumbering{arabic}
\ofoot{\pagemark}

\section{Einf\"uhrung}
% Basic Introduction, Goal of the Project and very short description of the system (see section ...)
\label{sec:introduction}

\subsection{Einleitung}
%\input{}

\subsection{Aufgabenstellung}
%\input{}

\subsection{Ziele und Zielgruppen}
%\input{}

\subsection{Konzept}
%\input{}

\newpage %----------------------------------------------------------------------------------------------

\section{Grundlagen und vorhandene Technologien}
%\input{}

\newpage %----------------------------------------------------------------------------------------------
\subsection{Methodiken zur Verbreitung}
%\input{}
\newpage

\subsubsection{Citizen Science}
%\input{}

\newpage
\subsubsection{Blidungssektor}
%\input{}

\subsection{Vorhandene Aquaponiksysteme}
%\input{}

\newpage
\subsubsection{Grove Ecosystem}
%\input{}

\subsubsection{EcoQube C}
%\input{}

\subsubsection{Ponix Systems}
%\input{}



\newpage %----------------------------------------------------------------------------------------------
\section{Projektmanagement}
%\input{}

\subsection{Projektmanagement Methode}
%\input{}

\subsection{Teamstruktur}
%\input{}

\subsection{Aufgabenteilung}
%\input{}

\subsection{Terminplanung}
%\input{}

\subsection{User Stories}
%\input{}

\subsection{Sprint Dokumentation}
%\input{}

\newpage %----------------------------------------------------------------------------------------------
\section{Evaluierung der benötigten Technologien und Komponenten}
%\input{}

\subsection{Single Board Computer}
%\input{}

\subsection{Sensoren}
%\input{}

\subsection{Aktoren}
%\input{}

\subsection{Datenbankmanagementsystem}
%\input{}

\subsection{Webframework}
%\input{}

\subsection{Daten\"ubertragung}
%\input{}

\newpage %----------------------------------------------------------------------------------------------
\section{Projekt Umsetzung - Designunterlagen}
%\input{}

\subsection{Use-Case Diagramm}
%\input{}

\subsection{Datenakquisition}
%\input{}

\subsubsection{Technologie}
%\input{}

\subsubsection{Ablauf}
%\input{}

\subsection{Datenverarbeitung}
%\input{}

\subsubsection{Technologie}
%\input{}

\subsubsection{Umsetzung des Ablaufs}
%\input{}


\newpage %----------------------------------------------------------------------------------------------
\section{Userinterface des Aquaponik Systems}
%\input{}

\subsection{Frontend}
%\input{}

\subsection{Backend}
%\input{}

\newpage %----------------------------------------------------------------------------------------------
\section{Ausfallsicherheit und Konsistenzsicherung der Daten}
%\input{}

\subsection{Zwischenspeicherung beim Aquaponik System}
%\input{}

\subsubsection{Ausfall der Internetverbindung}
%\input{}

\subsubsection{Ausfall der Stromversorgung}
%\input{}

\newpage %----------------------------------------------------------------------------------------------
\section{Ausblick}
%\input{}

\newpage %----------------------------------------------------------------------------------------------
\cfoot{}
\pagenumbering{roman}
\section{Appendix}
\label{sec:appenix}

\subsection{Glossaries}
\label{subsec:glossaries}
\begingroup
\renewcommand{\section}[2]{}
\printglossary[style=tree]
\endgroup
\newpage

{\small\color{white}.}
\vspace{-2cm}
\subsection{Figures}
\label{subsec:figures}
\begingroup
\renewcommand{\section}[2]{}
\listoffigures
\endgroup

\subsection{Listings}
\label{subsec:listings}
\begingroup
\renewcommand{\section}[2]{}
\lstlistoflistings
\endgroup

\subsection{Sources}
\label{subsec:sources}
\begingroup
\renewcommand{\section}[2]{}
\bibliographystyle{alpha}
\bibliography{sources}
\endgroup


\end{document}
