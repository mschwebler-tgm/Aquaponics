\cfoot{Samuel Schober}

\subsubsection {Industrielle Aquaponikanlagen}
Zu den größten Aquaponikanlagen, welche derzeit im Einsatz sind, zählen auch diverse Gewächshäuser. Diese züchten nachhaltig Nutzpflanzen und Fische um sie dann an lokalen Märkten anzubieten. Als eines der besten Beispiele kann die ECF Farm Berlin \cite{ECF}, eine Aquaponik Farm in Berlin \cite{Aquaponik_Anlage_Wien} genannt werden, die lokale Märkte und Restaurants mit Pflanzen und Fischen versorgt. Weiters ist ein Projekt einer ähnlichen Aquaponik Farm in Wien, in der Nähe des Wiener Hauptbahnhofs zu nennen, das erbaut und betrieben werden soll.\\ \mbox{} \\
Der Zuchtprozess wird von der Pflanzenart bestimmt, die produziert werden soll. Pflanzen, welche tiefer wurzeln und mehr Halt am Wurzelwerk brauchen, werden in einem Media Filled Bed gezüchtet. Ein solches Beet ist zumeist mit Blähton Kügelchen gefüllt. Diese Art des Aquaponik-Anbaus bringt zwar den Vorteil dass man mehrere Pflanzenarten als bei den anderen beiden Systemen anbauen kann. Es hat jedoch den Nachteil, dass diese Systeme betreffend der Platznutzung weniger effizient sind. \\ \mbox{} \\ Die anderen beiden Ausführungen eines Aquaponiksystems sind erstens die Nutrient Film Technique (\gls{NFT}) hierbei werden die Pflanzen in einem Wachs Medium aufgezogen, welches wiederum in einen NFT-Kanal steckt. Dabei handelt es sich um einen Kunststoffkanal, in welchem das Nährstoffhaltige Wasser fließt. Als Zweites ist die Deep Water Culture (\gls{DWC}) zu erwähnen. Hierbei werden die Pflanzen in einem, auf dem nährstoffhaltigem Wasser schwimmenden, Floating Raft angebaut, sodass die Wurzeln in einer individuell eingestellten Tiefe in das Wasser hängen. \gls{DWC} ist eine der am meist genutzten Arten von Aquaponik im industriellen Sektor und zeichnet sich dadurch aus dass sie genau so platzeffizient ist wie \gls{NFT}, aber im Gegensatz dazu auch erlaubt Pflanzen anzubauen, die längere Wurzeln haben.

\subsubsection{Forschungssektor}
Bisher hat die Forschung sich vor allem auf Manipulationen (z.B. Bestrahlung durch verschiedene Lichtfrequenzen) zur Verstärkung gewisser Pflanzeneigenschaften konzentriert. Aquaponik bietet die Möglichkeit mit derselben Pflanze in kürzerer Zeit mehr Ertrag zu erwirtschaften. Ähnliche Versuche gibt es in der Genmanipulation. Anders als in diesem Bereich, dessen längerfristige Folgen noch nicht absehbar sind, kann Aquaponik den Ertrag der Pflanze, wie sie von Natur aus ist, vervielfachen.\\
Durch kostengünstige Aquaponik- oder Hydroponiksysteme, mit dynamischen Beleuchtungssystemen wie das LED-Board von Ponix Systems können mit Hilfe von Erfahrungsberichten der Nutzer, die Effekte verschiedener Lichtfrequenzen auf diverse Pflanzen gesammelt werden. Einerseits wird das für die Züchtung neuer Arten dieser Pflanzen genutzt, andererseits können diese Erfahrungsberichte in der Wissenschaft archiviert und in anderen Zusammenhängen wiederverwertet werden.\\ \mbox{} \\
Als Methode der Einbindung von Bürgern in wissenschaftliche Gebiete, wie die Erforschung der Effekte der Bestrahlung mit verschiedenen Lichtfrequenzen auf Pflanzen, gibt es Citizen Science.\\ \mbox{} \\ Citizen Science, eine Form des wissenschaftschaftlichen Arbeitens, bei dem Bürger in wissenschaftliche Projekte eingebunden werden. Citizen Science Projekte werden in vielen Forschungsgebieten angeboten. Messungen und die Einreichung bzw. Auswertung von Daten sind dabei die Aufgabengebiete, die zumeist von Interessierten ausgeführt werden. Diese Aufgaben wären genau diejenigen, welche für die Erforschung der Einflüsse durch verschiedene Lichtfrequenzen auf verschieden Obst- und Gemüsesorten am ehesten benötigt würden. \\ \mbox{} \\ Ein solches Projekt könnte in Zusammenhang mit den oben genannten unbekannten Möglichkeiten der Pflanzenzüchtung bzw. -manipulation angestrebt werden.  Als Vorteile eines Citizen Science Projekts sind folgende Punkte erwähnenswert:
\begin{itemize}
    \item{Erhöhte Messdatenanzahl}
    \item{Schnellere Verarbeitung von großen Mengen an gesammelten Daten}
    \item{Steigerung des Interesses an der Wissenschaft}
\end{itemize}

\subsubsection{Homeponic}
Ein Aquaponiksystem im Haushalt bringt folgende Vorteile:
\begin{itemize}[leftmargin=4.5mm]
    \item \textbf{Saisonale Unabhängigkeit}\mbox{} \\
    Durch die Ermöglichung eines konstanten Klimas (in einem wärmegedämmten Haus) und die dauerhaft gleiche Lichteinstrahlung auf die Pflanzen, kann über das ganze Jahr jede Pflanze angebaut werden. Vorraussetzung ist natürlich die Kompatibilität mit einem Aquaponiksystem. Somit müssen saisonale Obst- und Gemüsesorten nicht von anderen Bezugsquellen importiert werden und Abgase können durch den sonst benötigten Transport reduziert werden. 
    \item \textbf{Obst- und Gemüsesorten kostengünstiger beziehen}\mbox{} \\
    Da nur Kosten für das Saatgut entstehen, und diese im Vergleich zu den Kosten der ausgewachsenen Pflanzen nur einen Bruchteil ausmachen, können Kosten reduziert werden. Der durchschnittliche Inhalt an Samenkörner einer Tomatensaatgut-Packung liegt bei 10 Stück und kann bei herkömmlichen Cherrytomaten zwischen 2,50€ und 4€ kosten. Eine bereits gezogene Pflanze der gleichen Tomatensorte kann im Supermarkt bis zu 3,50€ kosten. Außerdem trägt die Tomatenpflanze aus dem Supermarkt weniger Früchte, als eine im Homeponic gezüchtete Pflanze, die unter den dort herrschenden optimalen Bedingungen eine Maximalanzahl an Früchten hervorbringen kann.
    \newpage
    \item \textbf{Erhöhte Effizienz des Pflanzenanbaus}\mbox{} \\
    Pflanzen, welche mittels Aquaponik angebaut werden, wachsen bis zu 75\% schneller. Diese erhöhte Wachstumsgeschwindigkeit liegt darin begründet, dass die Pflanzen dauerhaft mit nahezu perfekten Umständen umgeben sind. Sie werden über das Wasser mit Nährstoffen versorgt, durch spezielle LED Boards wird ermöglicht, dass genau die Lichtfrequenzbereiche, welche die Pflanzen zum Wachsen brauchen in regelmäßigen Zeitintervallen auf die Pflanzen gestrahlt werden. Die Pflanzen wachsen daher immer in einem annähernd optimalen Klima.
\end{itemize}