\subsection{Aquaponik in Österreich}
In Österreich sind Aquaponik Systeme nicht weit verbreitet. Sie finden Anwendung in Bereichen wie dem industriellen Anbau von Nutzpflanzen. Daher sind sie immer in großen Maßstäben angelegt und für den Privatgebrauch nicht praktikabel. Urban Green entwickelt mit Homeponic erstmals ein System, das auch im kleinen Maßstab funktionstüchtig ist.
\subsubsection{Ponix Systems}
Der Projektpartner Ponix Systems \cite{PonixSystems} ist ein Start-up-Unternehmen, welches ein kompaktes Hydroponiksystem namens "`Herbert"' entwickelt hat. Die Produktion dieses Hydroponiksystems wurde mittels Crowd Funding finanziert und holte damit über 200.000 € ein. Als Alleinstellungsmerkmal zeichnet sich Ponix Systems beispielsweise durch ihr eigenentwickeltes \gls{LED}-Pflanzenlicht aus, das ein optimales Lichtverhältnis für die Pflanzen bietet.
\subsubsection{Aquaponic Austria}
Die Aquaponic Austria \cite{AquaponikAustria} ist ein Verein für Aquaponik Begeisterte. Es postet Neuigkeiten über Aquaponik in Österreich auf ihrer Website und bietet unter anderem auch ein Forum, bei welchem man sich über Fragen bezüglich Aquaponik informieren kann, an.