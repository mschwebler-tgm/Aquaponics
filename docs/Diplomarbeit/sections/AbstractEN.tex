\section*{Abstract}
\cfoot{Ramin Bahadoorifar}

This diploma project is about the development of an automated low-cost aquaponic-system prototype in cooperation with the startup company Ponix Systems. An aqua-ponic-system is a closed ecosystem which consists of an aquarium and a flower bed on top of it. The water in the aquarium gets enriched with various minerals by fish-wastes and is pumped to the flower field. With these essential nutrients and the help of a special illumination the crops' growth time can be shortened significantly. \\ \mbox{} \\
The main goal of this project is to create an automated aquaponic product that contains various additional features to help newcomers to get familiar with the world of aquaponic. In particular we implemented sensores (pH-, EC- and temperature sensor) and actuators (fish feeder and a specialised illumination) to allow the user to automate the system as far as possible. Furthermore a touchscreen was installed to be able to show real-time data to the user and to establish a connection to the local Wifi. More specifically the sensor-data is temporarely saved to an in-memory database (Redis) on a raspberry pi to be accessed by the webapp later on. \\ \mbox{} \\
In order to create statistics and trendanalyses the data is sent to a centralised server where it gets persisted. This server also offers a webapp where all data can be reviewed and analysed by the user. Via both webapps (www and local) it is possible to monitor and controll the aquaponic system. The difference is that the centralised webapp (which can be accessed over the internet) supports visualisation of longtermdata, whereas the raspberry pi is not suitable to for this kind of feature. Moreover the user can monitor and adjust his aquaponic system via the world wide web and one doesn't have to be present to regulate parameters. \\ \mbox{} \\
The web application was developed with the innovative framework Angular 2 which uses WebSockets for communication between server and webapp. Node.js was used as a server framework both on the central server and on the aquaponic-system. \\ \mbox{} \\
By automatically creating regular backups to prevent dataloss after a power failure and saving timestamps to the corresponding data to transmit it after the internet connection gets established again, the system is safe against power- and Wifi failures. \\ \mbox{} \\
The final stage of the project is to build a prototype which fits the requirements to bridge the current gap in the market of middle europe.