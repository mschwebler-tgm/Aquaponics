\cfoot{Matthias Schwebler}
\setcounter{page}{7}
Betreffend Systemen, die sich mit der Aufzucht von Nutzpflanzen auf Basis eines nähr-stoffliefernden Wasserbeckens beschäftigen, sind vor allem die Firmen ECF-Farmsystems, Lets-Grow und Fish A Plant zu erwähnen. \cite{fishAPlant, letsGrow, ECF} \\ 
Aquaponik Systeme sind heutzutage haupts\"achlich im gr\"o{\ss}eren Ma{\ss}stab bei der Aufzucht von Nutzpflanzen im Einsatz. F\"ur den normalen Haushalt gibt es allerdings bis jetzt keine marktf\"ahige L\"osung. Die zwei gr\"o{\ss}ten Crowdfunding Projekte sind EcoQube C und Grove Ecosystem die nachfolgend umrissen werden.
\subsubsection{EcoQube C}
Der EcoQube C vereint ein handliches Aquaponik System mit elegantem Design, jedoch mangelt es an Konfigurationsmöglichkeiten. Es sind ein paar Pflanzen und Fische vorgesehen, die verwendet werden sollen. Auch liegt das Fassungsverm\"ogen des Aquariums unter 50 Liter \cite{EcoQube}, was zur Folge hat, dass in \"Osterreich maximal ein einziger Fisch darin gehalten werden kann. Dies ist im Tierschutzgesetz § 7 Abs 9 spezifiziert: "`... Eine dauerhafte Haltung auch kleiner Arten in Aquarien unter 54 Liter ist verboten"' \cite{Zierfischforum, TuM}. \\
\subsubsection{Grove Ecosystem}
Das Grove Ecosystem ist das umfangreichste System, wenn es um Aquaponik Systeme im Haushalt geht. Es bietet  Sensoren (Luftfeuchtigkeit und -temperatur sowie Wasserstand und -temperatur), welche \"uber eine App abgefragt werden k\"onnen. Diese bietet zus\"atzlich eine gro{\ss}e Ansammlung an Daten und Empfehlungen f\"ur m\"ogliche Fische und die dazu passenden Pflanzen. Dieses Paket ist allerdings nur in den USA und Kanada, mit einem Einstiegspreis von ca 4000\euro \: erh\"altlich. \cite{Grove}\\
Zusammenfassend kann gesagt werden, dass beide Systeme in \"Osterreich kaum brauchbar bzw. nicht erh\"altlich sind. Bei Homeponic wird Wert darauf gelegt, dass das fertige Produkt f\"ur jeden leistbar ist, indem Features weggelassen werden, welche nicht unbeding ben\"otigt werden. Au{\ss}erdem wird \"au{\ss}erst stromsparende Hardware verwendet, um so wenig monatliche Kosten wie m\"oglich zu verursachen. \\ 
