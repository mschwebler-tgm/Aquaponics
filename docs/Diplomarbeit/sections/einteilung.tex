\cfoot{Team}
Aus der Analyse der Ziele ergaben sich folgende Arbeitsaufteilungen auf die Teammitglieder.
\begin{table}[ht]
\centering
\begin{tabular}{|l|l|lll}
\cline{1-2}
\textbf{Name} & \textbf{Hauptaufgabenbereich} &  &  &  \\ \cline{1-2}
Samuel Schober & CD (1), Aufbau des System-Prototyps (2) &  &  &  \\ \cline{1-2}
Ramin Bahadoorifar & Webapp (3), Backendanbindung (4)&  &  &  \\ \cline{1-2}
Matthias Schwebler & \begin{tabular}[c]{@{}l@{}}Sensordatenerfassung (5) und -verarbeitung (6), \\ Aktorensteuerung (7)\end{tabular} &  &  &  \\ \cline{1-2}
Konrad Kelc & Recherche (8), Datenbank &  &  &  \\ \cline{1-2}
\end{tabular}
\end{table}

\begin{enumerate}
    \item CD \\
    Erstellen eines Corporate Design: Logoentwicklung, Dokumentdesign
    \item Aufbau des System-Prototyps \\
    Aufbau des Prototyps: Zusammenbau und Modifizierung des Aquarium Aufsatzes, dauerhafte Überwachung des laufenden Systems
    \item Webapp \\
    Anzeigen der Sensordaten, sowie das Ansteuern der Aktoren auf einer Weboberfläche
    \item Backendanbindung \\
    Kommunikationsschnittstelle zwischen Webapp und dem Zentralserver
    \item Sensordatenerfassung \\
    Ansprechen der folgenden Sensoren: Temperatur-, pH-Wert- und EC-Wert-Sensor
    \item Sensordatenverarbeitung \\
    Zwischenspeichern der Daten und eventuelle Warnung bei zu hohen bzw. zu niedrigen Werten
    \item Aktorensteuerung \\
    Ansprechen der folgenden Aktoren: Pflanzenbeleuchtung, Futterautomat und Peristaltikpumpe
    \item Recherche \\
\end{enumerate}