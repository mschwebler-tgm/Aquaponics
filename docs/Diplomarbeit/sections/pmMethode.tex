\cfoot{Samuel Schober}
\setcounter{page}{19}
Zu Beginn des Diplomprojektes musste die Entscheidung gefällt werden, ob eine Methode des agilen oder eine Methode des klassischen Projektmanagement Stils
verwendet wird.\\ \mbox{} \\
Grob gesagt liegt der Unterschied der beiden Stile darin, dass agile Methoden flexibler gestaltet sind, sodass Änderungen am Projekt ständig integriert und ohne größere Beschwerden durchgesetzt werden können. Bei klassischen Projektmanagement Methoden können Änderungen des Projektes (bzw. des Projektumfangs) schwerwiegend und wegen des statischen Projektumfangs nur schwerlich umgesetzt werden. \\ \mbox{} \\
Nach einer Besprechung fiel die Entscheidung aufgrund folgender, für das Projekt wesentlicher Vorteile auf die agile Methode Scrum. 
\begin{description}
     \item[Flexibilität bei Änderungen]\mbox{} \\
    Scrum stellt alle Projektanforderungen als User Storys dar. Sie können jederzeit durch Change Request geändert werden. Durch die generelle Bereitschaft Änderungen in die Arbeitsweise zu inkludieren, werden Lösungen für unvorhergesehene Probleme leichter in das Projekt integrierbar.
     \item[Regelmäßige Abnahmen durch den Auftraggeber]\mbox{} \\
    Bei Srum werden User Storys in einen übergeordneten Sprint eingeteilt, welcher wiederum einem Release zugehörig ist. Sprints sind Teile eines Projekts, welche sequenziell abgearbeitet werden und vom Aufgabenumfang so gewählt sind, dass sie bis zu einem festgelegten Datum machbar sind. Somit wird durch die einzelnen Sprintabnahmen ermöglicht, dass der Auftraggeber bei möglichen Änderungswünschen rechtzeitig eingreifen und seine Vorstellungen in das Produkt (bzw. das Projekt) einbringen kann, sodass diese Änderungen zeitgemäß eingeplant und erfüllt werden können.
     \item[Bestehende Kommunikation]\mbox{} \\
    Jeder Arbeitstag bei Scrum wird gestartet indem die Teammitglieder sich bei einem Standup-Meeting (Daily Scrum) treffen und besprechen was sie an diesem Tag erreichen möchten und was sie am vorherigen Arbeitstag erreicht haben. Weiters wird nach jeder Sprintabnahme ein retroperspektives Meeting abgehalten. Das Meeting bezweckt, dass jedes Teammitglied mitteilen kann wie es ihm/ihr ergangen ist, wo Verbesserungswünsche liegen bzw. was ihm/ihr gut gefallen hat.
\end{description}
\mbox{} \\
All diese Punkte tragen dazu bei, dass eine fortwährende Kommunikation zwischen den Teammitgliedern und dem/den Stakeholder(n) Kunde(n) stattfindet und dass selbst bei Änderungen schnell eine Lösung gefunden werden kann und das Risiko der Entstehung weiterer Problemen, minimiert wird.