\cfoot{Samuel Schober}
Wegen dem derzeitigen Mangel an Heimlösungen, in Sachen Aquaponik-Systemen, soll das erste automatisierten Aquaponik-Heim-System für den europäischen Markt entwickelt werden. So hat der Nutzer die Möglichkeit, bei sich zu Hause, diverse Nutzpflanzen ökologisch anzubauen. Dabei wird ihm durch die Automatisierung des Systems auch noch einiges an Arbeit abgenommen. Dinge wie Fische füttern, regelmäßige pH-Wertmessungen und das Ein- und Ausschalten der Pflanzenbeleuchtung werden für den Nutzer, durch die Automatisierung des Systems, übernommen. Die Parameter für diese Aufgaben werden mittels einer Webapplikation an das System übermittelt.\\ \mbox{} \\
Durch die Verbreitung von Aquaponik-Systemen kann erreicht werden, dass weniger saisonale Gemüse, Früchte und Kräuter nach Österreich importiert werden müssen und dadurch der \ch{CO2} Haushalt gesenkt wird. Außerdem kann man diese Nutzpflanzen das ganze Jahr lang anbauen und sicher sein, welche Mittel auf die Nutzpflanzen angewendet wurden.\\ \mbox{} \\ 
Folgende Zielgruppen sollen mit dem Homeponic System angesprochen werden.
\begin{description}
	\item[Hobbygärtner]\mbox{} \\
Personen, die gerne selbst Nutzpflanzen anbauen und eine Möglichkeit suchen saisonale Früchte und Kräuter auch außerhalb der Saison anzubauen, generell eine andere Art der Pflanzenaufzucht suchen, oder sich für Aquaponik interessieren.
	\item[Städter]\mbox{} \\
Personen, welche selbst gerne Nutzpflanzen anbauen möchten aber keine Möglich- keit dazu haben. Das Homeponic System bietet die Möglichkeit auch ohne eigenen Garten Pflanzen zu züchten.
	\item[Aquaponik-Fans]\mbox{} \\
An Aquaponik interessierte Personen, die aus Platzmangel kein herkömmliches\\ Aquaponik-System betreiben können.
\end{description}
