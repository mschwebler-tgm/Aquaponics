\cfoot{Matthias Schwebler}
Die Anforderungen an das Projekt werden mithilfe von User Stories definiert. Die Folgenden umreißen die Grundanforderungen an das Aquaponik-System in groben Zügen, für die internen Entwicklungen noch einige weitere User Stories definiert wurden. 

\userstory{Benutzer}{1}{ein fertiges Produkt mit einfacher und intuitiver Bedienung erhalten}{dadurch die Inbetriebnahme auf Anhieb funktioniert, und ich nicht viel konfigurieren muss.}{Das System soll nicht nur für technikaffine Personen bedienbar sein.}

\userstory{Benutzer}{2}{möglichst wenig mit dem Produkt	interagieren müssen}{ich nicht immer daran denken will und auch gefahrlos auf Urlaub fahren will}{Das in Betrieb genommene System soll soweit automatisiert laufen, dass der Benutzer nicht regelmäßig Wartungsarbeiten erledigen muss.}

\userstory{Benutzer}{3}{den derzeitigen Status meines Aquaponik Systems von überall abrufen können}{ich etwaige Konfigurationen anpassen möchte um den Status meines Systems zu verbessern}{Das System soll von außerhalb überwacht und gesteuert werden können}

\userstory{Benutzer}{4}{Vorschläge für eine sinnvolle Besetzung des Aquariums und des Beetes bekommen}{ich als Einsteiger im Aquaponik Bereich wenig Erfahrung mitbringe.}{Der Zugriffspunkt auf das System (Webapp) bietet die Möglichkeit Konfigurationen für Fisch- und Pflanzenbesetzung einzusehen und direkt zu übernehmen.}

\userstory{Fisch}{5}{regelmäßig gefüttert werden}{ - (Grundanforderung)}{Der Futterautomat ist automatisiert. Parameter wie Fütterungszeiten und -menge können über die Webapp angepasst werden}

\userstory{Pflanze}{6}{dass Nährstoffe automatisch zugeführt werden}{so die Wachstumsdauer verringert wird}{Die Wachstumsdauer der Pflanze ist im Vergleich zum normalen Anbau geringer}

\userstory{Pflanze}{7}{eine adäquate Beleuchtung (Lichtintensität + Wellenlänge + Dauer)}{so die Wachstumsdauer weiter reduziert wird.}{Die Wachstumsdauer der Pflanze ist im Vergleich zum normalen Anbau geringer.}