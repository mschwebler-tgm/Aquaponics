\section*{Kurzfassung}
\setcounter{page}{5}
\cfoot{Matthias Schwebler}
 Die vorliegende Arbeit beschäftigt sich mit der Entwicklung eines Low-Cost Aquaponik-Systems. Es wird in Zusammenarbeit mit der Firma Ponix Systems ein Prototyp für den Heimgebrauch entwickelt.\\ Damit folgen wir dem globalen Trend im bewussten, ressourcenschonenderen Umgang mit der eigenen Lebensumwelt. In den letzten Jahren hat sich der Trend des "`urban gardening"' verstärkt. Menschen, die keinen Zugang zu Gemeinschaftsgärten oder anderen Freiräumen haben, können mithilfe von \textit{Homeponics} im eigenen Zimmer einen kleinen Beitrag zur Selbstversorgung leisten. Durch Urban Green und das entwickelte Homeponic-System kann die derzeitige Marktlücke von Aquaponik Systemen für den Hausgebrauch in Mitteleuropa gefüllt werden.\\ \mbox{} \\ 
 Ein Aquaponik-System ist ein geschlossenes Ecosystem, in dem Nutzpflanze und Aquariumbewohner eine Symbiose eingehen. Einerseits wird durch Bewässerung der Pflanzen mit Aquariumwasser und zusätzlicher, spezieller Beleuchtung eine stark reduzierte Wachstumszeit erzielt. Andererseits profitiert das Aquarium, weil Nährstoffe, die von den Fischen abgegeben werden und zu vermehrten Algenwuchs führen können, von den Pflanzen aufgenommen und verwertet werden. \\ \mbox{} \\ 
 Im Zuge dieses Projekts wurde ein pflegeleichtes \"Okosystem - bestehend aus einem Aquarium und einem Beet - entwickelt und auf \"Uberwachung und Automatisierung hin optimiert. Dazu wurden diverse Sensoren, Aktoren und Single Board Computer, welche die Regulierung der Parameter des Systems \"ubernehmen, verbaut.\\ \mbox{} \\ 
 Grob zusammengefasst setzt sich das Projekt aus drei Teilen zusammen: Dem Aquarium mit eingebauten Sensoren und Aktoren, dem SBC mit einem Touchscreen und der Webapplikation mit einer Datenbank im Hintergrund. \\ \mbox{} \\
Die von den Sensoren gemessenen Daten des Aquariums bzw. der Pflanzen werden an einen zentralen Server geleitet und dort persistiert um gegebenenfalls gew\"unschte Diagramme und Graphen zu erstellen. F\"ur die Persistierung wird auf Serverseite eine NoSQL Datenbank (MongoDB) verwendet und auf Seiten der Single Board Computer eine lightweight und schnelle In-Memory-Datenbank (Redis).\\ \mbox{} \\
Das Homeponic-System wird von einer Webapplikation aus beobachtet und gesteuert. Diese kann nicht nur durch einen kleinen Touchsreen bedient werden, sondern ist auch im Internet erreichbar. Die dafür benötigten Daten werden an einem Zentralserver durch ein Socket in einem bestimmten Interval gesendet. Sie werden zeitlich geordnet in einer NoSQL Datenbank gespeichert. Selbst bei Internetunterbrechungen zeichnet das System weiterhin Daten auf und sendet sie bei erneutem Internetzugang mit korrektem Zeitstempel an den Server. So hat der Nutzer den Vorteil, dass er sich nicht täglich mit der Pflege des Aquariums beschäftigen muss. Gleichzeitig kann er aber den Überblick über die Entwicklung aller Werte behalten. \\ \mbox{} \\ 

\afterpage{\blankpage}