\section*{Kurzfassung}
\cfoot{Matthias Schwebler}

Dieses Diplomprojekt, wird in Kooperation mit der Firma "Ponix Systems" durchgef\"uhrt und handelt \"uber die Entwicklung eines Low-Cost Prototypen eines Aquaponik Systems f\"ur den Heimgebrauch um dem globalen Trend der Umweltverbesserung zu folgen. Bei einem Aquaponik System handelt es sich um ein in sich geschlossenes Ecosystem, in dem Nutzpflanzen mit Wasser aus einem Aquarium bewässert und zusätzlich mit spezieller Beleuchtung  behandelt werden, um eine stark reduzierte Wachstumszeit zu erzielen. Hauptziel unseres Projektes ist es, die derzeitige Marktl\"ucke von Aquaponik Systemen in Mitteleuropa zu f\"ullen. Daf\"ur wird ein pflegeleichtes \"Okosystem - bestehend aus einem Aquarium und einem Beet - entwickelt und auf die \"Uberwachung und Automatisierung optimiert. Dazu werden diverse Sensoren, Aktoren und Single Board Computer verbaut, welche die Regulierung der Parameter des Systems \"ubernehmen. Die Daten des Aquariums bzw. der Pflanzen werden an einen zentralen Server geleitet und dort persistiert um gegebenenfalls gew\"unschte Diagramme und Graphen zu erstellen. F\"ur die Persistierung wird auf Serverseite eine NoSQL Datenbank (MongoDB) verwendet und auf Seiten der Single Board Computer eine lightweight und schnelle In-Memory-Datenbank verwendet (RedisDB). 
Das Aquaponiksystem kann von einer Webapplikation aus beobachtet und gesteuert werden. Die dafür benötigten Daten werden an einem Zentralserver durch ein Socket in einem bestimmten Interval gesendet. Diese werden zeitlich angeordnet in der NoSQL Datenbank gespeichert und dem jeweiligen Benutzer zugeordnet, aus welchen man dann eine Zeitanalyse erstellen kann in Form eines Diagramms. Bei der Entwicklung der Webapplikation wurde das innovative Framework Angular 2 verwendet.
Die Verbindung des \"Okosystems zum Internet wird von einem Single Board Computer \"ubernommen, der \"uber einen kleinen Touchscreen bedient werden kann. Selbst bei Internetunterbrechungen zeichnet das System weiterhin Daten auf und sendet sie bei erneutem Internetzugriff mit korrektem Zeitstempel an den Server