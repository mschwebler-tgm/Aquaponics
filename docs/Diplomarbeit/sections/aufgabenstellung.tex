\cfoot{Samuel Schober}

Umzusetzen war ein automatisiertes Aquaponik-System mit allen benötigten Aktoren (Wasserpumpen, Wärmestab, Pflanzenbeleuchtung, Futterautomat) und Sensoren (Temperatursensor, pH- und EC-Sensoren). Ebenso eine dazugehörige Webapplikation, welche alle durch die Sensoren erfassten Daten anzeigt, statistisch auswertet und die Aktoren den Benutzereingaben entsprechend ansteuert.\\ \mbox{} \\
Als Grundprinzipien für das im Diplomprojekt entwickelte Aquaponik-System waren die folgenden Faktoren zu beachten:\mbox{} \\
\begin{description}
    \item [Low Cost]\mbox{} \\ Das Aquaponik-System soll für eine große Menge an potentiellen Nutzern erschwinglich sein. 
    \item [Low Maintenance]\mbox{} \\ Die Nutzer sollen sich nicht um alle Detaileinstellungen selbst kümmern müssen.
    \item [Usability]\mbox{} \\ Die Verwendung des Aquaponik-Systems soll so einfach wie möglich gestaltet sein.
\end{description}

Ein Schwerpunkt lag in der Evaluierung der Sensoren, welche so präzise wie nur möglich bei gleichzeitig niedrigen Kosten sein sollen, und der Aktoren, welche den Anforderungen entsprechend ansteuerbar bzw. programmierbar sein müssen. Am schwersten musste hierbei die Präzision des pH-Sensors gewichtet werden. Er muss eine sehr hohe Messgenauigkeit besitzen, damit die Fische keinem zu säurehaltigen, oder basischen Wasser ausgesetzt sind und daraus folgend keinem Stress, und nicht der Gefahr zu sterben ausgesetzt sind. Ebenso benötigen die Nutzpflanzen einen, auf sie angepassten, pH-Wert im Wasser, damit sie nicht verenden. Es gilt also mit einem Messgerät zwei verschiedene Bedürfnisse zu berücksichtigen.