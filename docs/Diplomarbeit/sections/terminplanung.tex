\cfoot{Samuel Schober}
Nachdem auf der Diplomarbeitsdatenbank-Webseite Meilensteine angegeben wurden, ergab eine Besprechung nach der Evaluierung, dass diese Meilensteine nicht ganz korrekt in unseren Zeitplan passen und wurden somit angepasst.

Das Resultat dieser Besprechung ergab den folgenden Terminplan:
\begin{table}[ht]
\centering
\label{my-label}
\adjustbox{max width=\textwidth}{%
\begin{tabular}{|l|l|l|}
\hline
\textbf{Termin} & \textbf{Status}                                                                                                                    & \textbf{Sprint(s)}                                               \\ \hline
12.11.2016      & Fertigstellung der Evaluierung                                                                                                     & Evaluation \#0                                                   \\ \hline
04.12.2016      & Alpha Release der Sensorauswertung und Testung                                                                                     & Sprint \#1                                                       \\ \hline
21.12.2016      & \begin{tabular}[c]{@{}l@{}}Alpha Release der Aktorsteuerung und \\ Fertigstellung des Corporate Design\end{tabular}                & \begin{tabular}[c]{@{}l@{}}Sprint \#2,\\ Sprint \#3\end{tabular} \\ \hline
10.01.2017      & \begin{tabular}[c]{@{}l@{}}Erstellung und Testung eines Webapp-Prototypen,\\ Erstellung und Testung der Datenbank\end{tabular}     & Sprint \#4                                                       \\ \hline
18.02.2017      & \begin{tabular}[c]{@{}l@{}}Aufbau des kompletten Systems, Beta Release \\ der Aktorsteuerung und der Sensorauswertung\end{tabular} & \begin{tabular}[c]{@{}l@{}}Sprint \#5,\\ Sprint \#6\end{tabular} \\ \hline
15.03.2017      & \begin{tabular}[c]{@{}l@{}}Abschluss der Beta Testung, \\ Beginn der fortlaufenden Testung des Homeponics\\ Aquaponik-Systems\end{tabular} &                                                                  \\ \hline
\end{tabular}%
}
\end{table}

Die Termine wurden so gewählt, dass die kritischsten Sprints (\#1, \#2, \#4) zu Anfang des Projekts implementiert werden und somit die Testung dieser Sprints schon früh beginnen konnte. Nach dem Beta Release ergibt sich somit ein Zeitpuffer von 3 Wochen, in welchem Probleme behoben werden können.